% You should title the file with a .tex extension (hw1.tex, for example)
\documentclass[a4paper, 11pt]{article}

\usepackage{amsmath}
\usepackage{amssymb}
\usepackage{fancyhdr}

\usepackage[margin=1in]{geometry}

\newcommand{\question}[2] {\vspace{.25in} \hrule\vspace{0.5em}
\noindent{\bf #1: #2} \vspace{0.5em}
\hrule \vspace{.10in}}
\renewcommand{\part}[1] {\vspace{.10in} {\bf (#1)}}

\newcommand{\myname}{Natthakan Euaumpon}
\newcommand{\myemail}{natthakaneuaumpon@gmail.com}
\newcommand{\myhwnum}{2}

\setlength{\parindent}{0pt}
\setlength{\parskip}{5pt plus 1pt}
 
\pagestyle{fancyplain}
\lhead{\fancyplain{}{\textbf{HW\myhwnum}}}      % Note the different brackets!
\rhead{\fancyplain{}{\myname\\ \myemail}}
\chead{\fancyplain{}{ICCS313 }}

\begin{document}

\medskip                        % Skip a "medium" amount of space
                                % (latex determines what medium is)
                                % Also try: \bigskip, \littleskip

\thispagestyle{plain}
\begin{center}                  % Center the following lines
{\Large ICCS313: Assignment \myhwnum} \\
\myname \\
\myemail \\
October 2019 \\
\end{center}

\question{1}{Part1}
\part{a}\\
$2T(\frac{n}{3})+1$\\
$a=2, b=3, d=0$\\
$log_{b}a = log_{3}9$\\
$log_{3}9 > 0$\\
$T(n)=O(n^{log_{3}2})$

\part{b}\\
$5T(\frac{n}{4})+n$\\
$a=5, b=4, d=1$\\
$log{b}a = log{4}5$\\
$log{4}5 > 1$\\
$T(n)=O(n^{log_{4}5})$

\part{c}\\
$7T(\frac{n}{7})+n$\\
$a=7, b=7, d=1$\\
$log_{b}a = log_{7}7$\\
$1 = 1$\\
$T(n)=O(nlogn)$

\part{d}\\
$9T(\frac{n}{3})+n^{2}$\\
$a=9, b=3, d=2$\\
$log_{b}a = log_{3}9$\\
$2 = 2$\\
$T(n)=O(n^{2}logn)$

\part{e}\\
$8T(\frac{n}{2})+n^{3}$\\
$a=8, b=2, d=3$\\
$log_{b}a = log_{2}8$\\
$3 = 3$\\
$T(n)=O(n^{3}logn)$

\part{f}\\
$T(n-1)+2$\\
$(T(n-2)+2)+2$\\
$(T(n-3)+2)+2+2$\\
$(T(n-4)+2)+2+2+2$\\
.\\
.\\
.\\
$(T(0)+2)+2(n-1)$\\
T(n-1) run $n$ times so it is $2n$\\
$T(n)=O(n)$

\part{g}\\
$T(n-1)+n^{c}$\\
$=(T(n-2)+(n-1)^{c})+n^{c}$\\
$=(T(n-3)+(n-2)^{c})+(n-1)^{c}+n^{c}$\\
.\\
.\\
.\\
$=(T(0)+(n-(n+1))^{c})+2^{c}+...+n^{c}$\\
$=T(0)+1^{c}+2^{c}+...+n^{c}$\\
$T(n)=1^{c}+2^{c}+...+n^{c}, \forall n \geq 1$\\
$T(n)=n^{c+1}, \forall n \geq 1$\\
$T(n) \leq n^{c+1} when c = 1 and \forall n \geq n_{0} = 1$\\
$T(n)=O(n^{c+1})$

\part{h}\\
$T(n-1)+c^{n}$\\
$=(T(n-2)+c^{n-1})+c^{n}$\\
$=(T(n-3)+c^{n-2})+c^{n-1}+c^{n}$\\
.\\
.\\
.\\
$=(T(0)+c^{n(n-1)})+c^{2}+...+c^{n}$\\
$=T(0)+c^{1}+c^{2}+...+c^{n}$\\
$T(n)=c^{1}+c^{2}+...+c^{n}$\\
$cT(n)=c^{2}+c^{3}+...+c^{n+1}$\\
$cT(n)-T(n)=(c^{2}+c^{3}+...+c^{n+1})-(c^{1}+c^{2}+...+c^{n})$\\
$T(n)\cdot (c-1)=c^{n+1}-c$\\
$T(n)=\frac{c \cdot (c^{n}-1)}{c-1}$\\
$\lim_{n\to\infty} \frac{\frac{c \cdot (c^{n}-1)}{c-1}}{c^{n}}$\\
$\lim_{n\to\infty} \frac{c\cdot (c^{n}-1)}{c^{n}\cdot (c-1)}$\\
$\lim_{n\to\infty} \frac{c^{n+1}-c}{c^{n+1}=c^{n}}$\\
$=1$\\
$T(n)\in\Theta(c^{n})$, therefore $T(n)\in O(c^{n})$

\part{i}\\
$2T(n-1)+1$\\
$=2(2T(n-2)+1)+1$\\
$=2(2(T(n-3)+1))+2+1$\\
.\\
.\\
.\\
$=T(0)+1+2+4+...+2^{n-1}$\\
$T(n)=1+2+4+...+2^{n-1}$\\
$2T(n)=2+4+...+2^{n}$\\
$2T(n)-T(n)=(2+4+...+2^{n})-(1+2+4+...+2^{n-1})$\\
$T(n)=2^{n}-1$\\
$\lim_{n\to\infty} \frac{2^{n}}{2^{n}}$\\
$=1$\\
$T(n)\in\Theta(2^{n})$, therefore $T(n)\in O(2^{n})$\\

\part{j}\\
$T(n^{\frac{1}{2}})+1$\\
$=(T(n^{\frac{1}{4}})+1)+1$\\
$=(T(n^{\frac{1}{8}})+1)+1+1$\\
.\\
.\\
.\\
$=T(n^{\frac{1}{2^{k}}})+k$\\
$n$ need to be greater than or equal to $2$ to get $T(0)$.\\
So, $2 = n^{\frac{1}{2^{k}}}$\\
$\log_{n} 2 \frac{1}{2^{k}}$\\
$2^{k} = \frac{1}{\log_{n} 2}$\\
$k = \log_2 \frac{1}{log_{n} 2}$\\
$k = \log_2\log_2 n$\\
$T(n)=\log_2\log_2 n$\\
$\lim_{n\to\infty} \frac{\log_2\log_2 n}{\log_2\log_2 n}$\\
$=1$\\
$T(n)\in\Theta(\log_2^{2} n)$, therefore $T(n)\in O(\log_2^{2} n)$
\question{2}{Part2}
Algorithm A:\\
$T(n)=3T(\frac{b}{3})+O(n)$\\
By using master theorem:\\
$a=3, b=3, d=1$\\
$\log_{b} a = log_3 3 = 1$\\
$1=1$\\
$=O(n\log n)$
Algorithm B:\\
$T(n) = T(n-1)+O(n)$\\
$T(n-1)+O(n)$\\
$=(T(n-2)+O(n))+O(n)$\\
.\\
.\\
.\\
$T(n)=O(n^{2})$\\
Algorithm C:\\
$T(n)=2T(\frac{n}{3})+O(n^{2})$\\
By using master theorem:\\
$\log_{b} a = \log_{3} 2$\\
$\log_{3} 2 < 2$\\
$=O(n^{2})$\\
Algorithm D:\\
$T(n)=5T(\frac{n}{4})+O(n)$\\
By using master theorem:\\
$\log_{b} a = \log_{4} 5$\\
$\log_{4} 5 > 1$\\
$=O(n^{\log_{4} 5})$\\
We should use algorithm d as it has lowest upper bound of time complexity.
\question{3}{Problem3}
The function f(n) use n as an input and the loop will stop running when $n=1$\\
This program calls itself 3 times each take $\frac{n}{2}$\\
The time complexity for printing is $O(1)$\\
This means:\\
$T(n)=3T(\frac{n}{2})+O(1)$\\
By using master theorem:\\
$\log_{b} a = \log_2 3$\\
$\log_2 3 > 1$\\
$=O(n^{\log_{2} 3})$
\end{document}

